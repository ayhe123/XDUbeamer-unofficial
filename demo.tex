\documentclass{ctexbeamer}
\usepackage[blue]{XDUbeamer}

\title{XDUbeamer 使用示例}
\subtitle{在这里输入副标题}
\institute{在这里输入学院}
\author{在这里输入名字}

\begin{document}
\begin{frame}[plain] % 标题页要加 [plain]
    \titlepage
\end{frame}
\begin{frame}
    \frametitle{目录}
    \tableofcontents
\end{frame}
\begin{frame}[fragile] % 使用了 \verb 环境的页面要加 [fragile]
    \frametitle{设置章节}
    \begin{itemize}
        \item 用 \verb|\section| 设置章, 加上 \verb|\frame{\sectionpage}| 使得标题单独成页.
        \item 用 \verb|\subsection| 设置节, 加上 \verb|\frame{\subsectionpage}| 使得标题单独成页.
        \item 暂时不支持其他标题单独成页.
    \end{itemize}
\end{frame}
\section{第1章}
\frame{\sectionpage}
\subsection{第1节}
\frame{\subsectionpage}
\section{定理环境}
\begin{frame}
    \frametitle{定理环境}
    \begin{definition}
        \alert{定理环境} 包括定义, 定理, 证明等的环境.
    \end{definition}\pause % 用 \pause 实现类似于PPT动画的功能
    \begin{theorem}
        这是一个定理.
    \end{theorem}\pause
    \begin{proof}
        这是定理的证明.
    \end{proof}\pause
    \begin{example}
        这是一个例子.
    \end{example}
\end{frame}
\section{图像}
\begin{frame}
    \frametitle{插入图像}
    浮动体没有序号.
    \begin{figure}[htbp!]
        \centering
        \includegraphics[scale=.5]{img/logo_blue.png}
        \caption{这是一张图片}
    \end{figure}
    \begin{figure}[htbp!]
        \centering
        \includegraphics[scale=.5]{img/logo_blue.png}
        \caption{这还是一张图片}
    \end{figure}
\end{frame}
\section{表格}
\begin{frame}
    \frametitle{排版表格}
    \begin{table}
        \centering
        \begin{tabular}{*{7}{c}}
          \toprule
          \multirow{3}{*}{算法} & \multicolumn{6}{c}{数据集} \\
          \cmidrule{2-7}
          & \multicolumn{3}{c}{Iris} & \multicolumn{3}{c}{Sonar} \\
          \cmidrule{2-7}
          & 指标1 & 指标2 & 指标3 & 指标1 & 指标2 & 指标3 \\
          \midrule
          算法1 & 0.647 & 0.907 & 0.249 & 1.129 & 0.545 & 0.991 \\
          算法2 & 0.647 & 0.903 & 0.255 & 1.289 & 0.589 & 0.974 \\
          算法3 & 0.668 & 0.926 & 0.210 & 1.282 & 0.561 & 0.986 \\
          算法4 & 0.678 & 0.919 & 0.223 & 1.281 & 0.554 & 0.988 \\
          \bottomrule
        \end{tabular}
        \caption{两个数据集上的聚类结果}
      \end{table}
\end{frame}
\section{排版公式}
\begin{frame}
    \frametitle{公式}
    单行公式:
    \[\operatorname{DTFT}[x(n)]=\sum\limits_{n=\infty}^{\infty}x(n)e^{-j\omega n}.\]

    多行公式:
    \begin{align*}
        S' & =\dfrac{1}{N}(HXu)^TH(HXu) \\
        & =\dfrac{1}{N}(u^TX^TH^T)HXu \\
        & =\dfrac{1}{N}u^TX^THHXu \\
        & =\dfrac{1}{N}u^TSu.
    \end{align*}
\end{frame}
\section{排版源程序}
\begin{frame}[fragile]
    \frametitle{源程序}
    用 \verb|lstlisting| 环境来排版源程序, 支持语法高亮:

    \begin{lstlisting}[language=Python]
def print_i(n):
    for i in range(n):
        print(i)
    return

print("开始打印")
# print 10 times
print_i(10)
# print 100 times
print_i(100)
# print 1000 times
print_i(1000)
    \end{lstlisting}
    \end{frame}
\section{项目主页}
\begin{frame}[fragile]
    \frametitle{项目主页}
    \begin{itemize}
        \item 项目主页: \url{https://github.com/ayhe123/XDUbeamer-unofficial}
        \item 欢迎 Star
        \item 如果有想要添加的功能或者发现项目有问题, 都可以去发 Issue
    \end{itemize}
\end{frame}
\end{document}